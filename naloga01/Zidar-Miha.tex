\documentclass[a4paper,11pt]{article}
\usepackage{a4wide}
\usepackage{fullpage}
\usepackage[utf8x]{inputenc}
\usepackage[slovene]{babel}
\selectlanguage{slovene}
\usepackage[toc,page]{appendix}
\usepackage[pdftex]{graphicx} 
\usepackage{setspace}
\usepackage{color}
\definecolor{light-gray}{gray}{0.95}
\usepackage{listings} 
\usepackage{hyperref}
\renewcommand{\baselinestretch}{1.2} 
\renewcommand{\appendixpagename}{Priloge}

\usepackage{float}

\lstset{ 
language=Python,
basicstyle=\footnotesize,
basicstyle=\ttfamily\footnotesize\setstretch{1},
backgroundcolor=\color{light-gray},
}

\title{1. Domača naloga: Nekaj lastnosti podatkov o\\ razvrščanju člankov v tematske skupine}
\author{Miha Zidar (63060317)}
\date{\today}

\begin{document}

\maketitle

\section{Uvod}

V prvi doma"ci nalogi smo si pogledali en del podatkov iz tekmovanja \href{http://tunedit.org/challenge/JRS12Contest}{JRS 2012 Data Mining Competition}, ter posku"sali ven razbrati nekaj zna"cilnosti o samih podatkih. Pregledovali smo kako pogosto se kak"sni atributi nahajajo, kako redka je matrika atributov in "se nekaj podobnih zanimivosti.

\section{Rezultati}

\subsection{Koliko primerov in atributov vsebujejo podatki?}

Na"s manj"si nabor podatkov vsebuje 2000 primerov in vsak primer ima 10000 atributov. 

\subsection{Kakšnega tipa so atributi?}

Vsi atributi so predstavljeni kot cela "stevila.

\subsection{Kako redka je matrika oz. kak"sen dele"z njenih elementov ima vrednost razli"cno od 0?}\label{delez1}

V matriki ima 82423 elementov vrednost reazli"cno od ni"c, kar predstavlja 0.4121\% celotne matrike.

\subsection{Koliko atributov ima vrednost razli"cno od 0 za posamezen primer?}

"Stevilo atributov razli"cnih od 0 pri posameznih primerih se giblje med 21 in 79

\begin{figure}[H]
\begin{center}
\includegraphics[scale=0.5]{src/nonZeroAttr.pdf}
\caption{Porazdelitev "stevila neni"celnih vrednosti atributov na posamezen primer}
\label{slika1}
\end{center}
\end{figure}

\subsection{V koliko primerih atribut zavzame neni"celne vrednosti?}
Ze pri prvem uprasanju smo videlu da je stevilo elementov ki so razli"cni od 0 zelo majhno. To prikazuje tudi spodnji graf.

\begin{figure}[H]
\begin{center}
\includegraphics[scale=0.5]{src/nonZeroPrimeri.pdf}
\caption{Porazdelitev "stevila vrednosti posameznih atributov z Y osjo v logaritemski skali}
\label{slika2}
\end{center}
\end{figure}

\subsection{Koliko je vseh razli"cnih oznak (razredov) v podatkih?}

V podatkih je skupaj 82 razli"cnih oznak. Vsebovane so vse oznake od 1 do 83, z izjemo oznake 77.

\subsection{S koliko različnimi oznakami so označeni primeri?}

Posamezni primeri imajo od 1 do 13 oznak, najvec jih je med 1 in 4. 

\begin{figure}[H]
\begin{center}
\includegraphics[scale=0.5]{src/labelCountsMax.pdf}
\caption{Porazdelitev "stevila oznak za posamezen primer}
\label{slika2}
\end{center}
\end{figure}

\subsection{Na"stej 3 najbolj pogosta "stevila nenicelnih vrednosti atributov pri posameznih primerih.}

Najbolj pogosto "stevilo neni"celnih vrednosti ima primer 38 ki ima 106 neni"celnih atributov. Za njem se primera 35 in 37 z 105 neni"celnimi vrednosti atributov.


\subsection{Koliko primerov je takih ki imajo najve"c oznak, in koliko atributov imajo taki primeri?}

Tak je samo en primer, in ima 39 nenicelnih atributov. Nekaj takega bi lahko tudi pricakovali, saj pase prav lepo v povprecje stevila neniclenih atributov, in to da ima najvec oznak o"citno ni pogojeno z stevilom atributov razli"cnih od ni"c.

\subsection{Koliko atributov imajo primeri z najbolj pogostim stevilom oznak in koliko je takih primerov?}

Takih primerov je skoraj "cetrtina in sicer natanko 418. Tudi tu se "stevilo nenicelnih atributov lepo ujema z povprecjem. Med temi primeri, "ceprav jih je veliko, ne najdemo nobenega z zelo majhnim ali zelo velikim stevilom neni"celnih atributov. 



\begin{figure}[H]
\begin{center}
\includegraphics[scale=0.5]{src/commonAttr.pdf}
\caption{"Stevilo neni"celnih atributov za elemente z najbolj pogostim stevilom oznak}
\label{slika2}
\end{center}
\end{figure}

%\subsection{Kakšna je porazdelitev za atribut z največ neničelnimi vrednostmi?}\label{dod2}
%
%Atribut z največ neničelnimi vrednostmi je atribut št. 7887, ki ima 537 neničelnih vrednosti. Najbolj pogosta neničelna vrednost tega atributa je 34, ki se pojavi 39-krat.
%
%
%\begin{figure}[H]
%\begin{center}
%\includegraphics[scale=0.3]{img/mostDefinedAttr.pdf}
%\caption{Porazdelitev vrednosti za najbolj določen atribut}
%\label{slika2}
%\end{center}
%\end{figure}
%
%\subsection{Katera oznaka se pojavi največkrat?}
%Najbolj pogosta oznaka je ''40'', ki se pojavi pri 502 primerih.

\section{Izjava o izdelavi doma"ce naloge}
Doma"co nalogo in pripadaj"ce programe sem izdelal sam.

\end{document}
