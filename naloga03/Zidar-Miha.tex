% To je predloga za poročila o domačih nalogah pri predmetih, katerih
% nosilec je Blaž Zupan. Seveda lahko tudi dodaš kakšen nov, zanimiv
% in uporaben element, ki ga v tej predlogi (še) ni. Več o LaTeX-u izveš na
% spletu, na primer na http://tobi.oetiker.ch/lshort/lshort.pdf.
%
% To predlogo lahko spremeniš v PDF dokument s pomočjo programa
% pdflatex, ki je del standardne instalacije LaTeX programov.

\documentclass[a4paper,11pt]{article}
\usepackage{a4wide}
\usepackage{fullpage}
\usepackage[utf8x]{inputenc}
\usepackage[slovene]{babel}
\selectlanguage{slovene}
\usepackage[toc,page]{appendix}
\usepackage[pdftex]{graphicx} % za slike
\usepackage{setspace}
\usepackage{color}
\definecolor{light-gray}{gray}{0.95}
\usepackage{listings} % za vključevanje kode
\usepackage{hyperref}
\usepackage{endnotes}
\renewcommand{\baselinestretch}{1.2} % za boljšo berljivost večji razmak
\renewcommand{\appendixpagename}{Priloge}

\lstset{ % nastavitve za izpis kode, sem lahko tudi kaj dodaš/spremeniš
language=Python,
basicstyle=\footnotesize,
basicstyle=\ttfamily\footnotesize\setstretch{1},
backgroundcolor=\color{light-gray},
}

\title{Razvrščanje člankov v tematske skupine}
\author{Miha Zidar (63060317)}
\date{\today}

\begin{document}

\maketitle

\section{Uvod}

Cilj te doma"ce naloge je bil naresti algoritem, ki na podlagi u"cne mno"zice podatkov "cim bolj u"cinkovito napove pripadajo"ce razrede, za posamezne primere iz testne mno"zice podatkov.

\section{Metode}

\subsection{Ocenjevanje to"cnosti}

\subsubsection*{$k$-kratno pre"cno preverjanje}
Z metodo $k$-kratnega pre"cnega preverjanja u"cno mno"zico razdelimo na k enakih delov. Potem pa $k$-krat ponovimo u"cenje pri katerem sedaj testno mno"zico predstavlja en del razdeljene u"cne mno"zice, za u"cno mno"zico pa uzamemo preostale primere. Po $k$ ponovitvah imamo za vse ucne primere na voljo pravilne rezultate in na"se napovedi. Za ocenjevanje teh napovedi obstaja ve"c ocen, kot sta na primer AUC in F-ocena, slednjo pa smo tudi uporabili.


\subsubsection*{F-ocena}
To je eden od mo"znih na"cinov ocenjevanja napovedi. Sestavljena je iz ocen precision in recall, vendar odpravlja njune slabosti pri preve"c ali premalo napovedanih podatkih. Za kon'cno oceno napovedi smo vzeli povpre"cje F-ocen za vse primere.

\[avgFscore_i\  =\ \frac{ \sum^{N}_{i=1} Fscore_i }{N} \]

\[Fscore_i\  =\ 2\ \cdot \ \frac{precision_i\, \cdot \, recall_i }{precision_i\, + \, recall_i } \]

\[precision_i \ =\ \frac{|\,TocniRazredi\, \bigcap\, NapovedaniRazredi\, |}{|\,NapovedaniRazredi\, |} \]
\[recall_i \ =\ \frac{|\,TocniRazredi \bigcap NapovedaniRazredi\, |}{|\,TocniRazredi\, |} \]

\subsection{Napovedni modeli}

\begin{description}
\item[k-najbli"zjih sosedov] (k-nearest neighbors, v nadaljevanju KNN) je algoritem ki za dani tesni primer, iz u"cne mno"zice poi"s"ce k najbli"zjih sosedov in glede na njihovo pripadnost razredom utezeno (glede na razdaljo) napove razred testnega primera. Algoritem sem implemtiral tako da razdaljo ra"cuna tako, da i"s"ce samo podobnosti med pari. To se je izkazalo za dosti bolj"so re"sitev kot uporaba hamingove ali evklidske razdalje. 

\item[Naklju"cni gozd] (random forest, v nadaljevanju RF) je algoritem ki iz u"cne mno"zice podatkov zgradi $n$ odlo"citvenih dreves in sestavi napoved iz posameznih napovedi teh dreves. Tukaj sem uporabil RF iz knji"znice Orange z privzetimi nastavitvami za 100 dreves. Da sem pa RF pomagal, sem med tribute v testni in u"cni mno"zici dodal se binarne vrednosti teh atributov.

\item[Kombinacija: ] Posebaj pa sem probaval tudi razlicne metode kombiniranja rezultatov pridobljenih iz zgoraj navedenih metod. Tukaj sem probal dva pristopa. Prvi je bil kombiniranje kon"cnih rezultatov, drugi in bolj"si pristop pa je bil kombiniranje verjetnosti, ki jih posamezna metoda vrne, ter nato sestavljanje kon"cnih rezultatov. \\

\end{description}

\subsubsection*{Kalibracija in izbor podatkov}
Glavni del, pa ni bil optimizacija ali kalibracija algoritmov, vendar izbiranje rezultatov za vsak algoritem. Temu sem tudi namenil najve"c "casa. Za kalibracijo "cim bolj"sega "stevila napovedanih razredov, sem prav tako uporabljla F-oceno z k-kratnim pre"cnim preverjanjem, vendar na mnogo manj"si mno"zici podatkov(1000 primerov in manjk kot 400 atributov). \\

\subsubsection*{Koda modelov}

Omeniti moram da sem zdru"zevanje rezultatov delal ro"cno v konzoli, vendar funkcija je vseeno zapisana v pomozni datoteki \textit{data.py} Napovedna modela se nahajata v datotekah z istim imenom (\textit{knn.py} za k najblji"zjih ter \textit{rf.py} za random forest). Zraven pa sem naredil "se nakaj skript ki so mi pomagale z delom z podatki, kot so branje, normalizacija tabel, zdru"zevanje in podobno. Te stvari se nahajajo v datoteki \textit{data.py}, skripta z katero sem pa dodal nove attribute v binarnih vrednostih je pa \textit{newAttrs.py}.


\newpage

\section{Rezultati}

Na tekmovalnem stre"zniku sem pod imenom \textit{zidarsk8} odajal svoje napovedi. Napovedi za k najbli"zjih sosedov sem pred tem "se lokalno testiral z pre"cnim preverjanjem, vendar naklju"cnih gozdov pa se zaradi po"casnosti nisem odlo"cil testirati lokalno, zato tudi ne morem primerjati moje ocene in kon"cne ocene na stra"zniku.\\

\begin{minipage}{14cm}
\begin{tabular}{l|r|r|p{6cm}}
ime datoteke & F-ocena & moja ocena & opis uporabljene metode \\ 
\hline 
\hline 
*result1331479186.csv & 0.31053 & 0.40389 & samo kot zanimivost, dober rezultat z neveljavno datoteko\\ 
\hline 
*result1331479186.csv  & 0.41069 & 0.40389 & tokrat ista datoeka v pravilni obliki. Navadni knn, ki gleda le podobnosti in izbere najbolj pogoste razrede prvih 15 najblizjih primerov. \\ 
\hline 
*skupni.csv  & 0.42993 &  & 4 razli"cne rezultate (2 or RF, in 2 od KNN) sem enostavno zdru"zil skupaj, tako da sm upo"steval razrede, ki se v teh modelih pojavijo vsaj 2krat. \footnotetext{*Oddaje do 12. 3. 2012} \\ 
\hline 
rf\_top\_proc\_02\_30.csv & 0.43764 &  & navadni random forest iz Orange z 100 drevesi, in izbiranje atributov v odvisnosti od najbolj"sega v vrstici. Verjetnost izbire atributa linearno pada do 30. \\
\hline 
rf\_1332091593\_02\_30.csv & 0.44242 &  &  enako kot prej"snji le da sem pri gradnji drevesa dodal "se vse atribute binarizirane.\\
\hline 
knn\_rf\_1332111511.csv & 0.45409 &  & tukaj sem normaliziral in ute"zeno se"stel verjetnosti razredov, ki sem jih dobil od RF (0.65) in KNN (0.35). Rezultate sem z svojim izbiranjem dobil z parametri 0.22 in 30.\footnote{prvo "stevilo je pove koliko se mora verjetnost drugega razreda ujemati z najbolj"sim, da ga "se dodamo v izbor. Drugo "tevilo pa pove kako hitro se prvo "stevilo priblizuje 1. maximalno "stevilo dodanih razredov je torej 30*(1-0.22) pod pogojem da imajo vsi enako verjetnost.}
\end{tabular} 
\end{minipage}



\section{Izjava o izdelavi domače naloge}
Domačo nalogo in pripadajoče programe sem izdelal sam. 
\end{document}
